% !TEX root = ../Main.tex
% ==============================================================================
% Dritter Abschnitt: Dezentrale Organisation
% ==============================================================================

\newpage
\setcounter{chapter}{3}
\chapter*{Dritter Abschnitt: Dezentrale Organisation}
\addcontentsline{toc}{chapter}{Dezentrale Organisation}
\chaptermark{Dezentrale Organisation}
\vspace*{-1.5em}
\rule{\textwidth}{0.5mm}
\vspace*{-2em}

\subsection{Fachschaft und Fachschaftsvertretung}
Die Studierenden einer Fakultät bilden eine Fachschaft. In der Fachschaft wird eine Fachschaftsvertretung gebildet. Die Fachschaftsvertretung nimmt die fakultätsbezogenen Studienangelegenheiten und Aufgaben im Sinne des § 65 Absatz 2 LHG auf Fakultätsebene wahr. Falls das StuPa den jeweiligen Fachschaften ein Budget für das Haushaltsjahr zur Verfügung stellt, dürfen die Fachschaftsvertretungen über das, der Fachschaft zugeteilte Budget verfügen. Dies muss jedoch in Absprache mit ihren jeweiligen studentischen Fakultätsräten erfolgen. Bei unsachgemäßem Umgang des Budgets muss der AStA oder das StuPa eingeschaltet werden und die Verfügungstellung des Budgets neu überprüft werden.

\subsection{Zusammensetzung der Fachschaftsvertretung}
Die Fachschaftsvertretung setzt sich aus den gewählten Fachschaftsvertretungen sowie den studentischen Fakultätsratsmitgliedern, die der Fachschaftsvertretung kraft Amtes angehören, zusammen.

\subsection{Wahlen zu den Fachschaftsvertretungen}
Jede Fachschaft wählt nach § 10 Absatz 1 Fachschaftsvertretungen. Darüber hinaus werden nach § 10 Absatz 2 studentische Fakultätsräte einer jeder Fakultät gewählt.

\subsection{Fachschaftsvorsitzender}
\begin{enumerate}[label=(\arabic*)]
  \item Der Fachschaftsvorsitzende führt die laufenden Geschäfte der Fachschaft, bereitet die Beschlüsse der Fachschaftsvertretung vor und führt sie aus.
  \item Er wird von der Fachschaftsvertretung aus ihrer Mitte für die Dauer der Amtszeit gewählt. Für die Wahl ist die Mehrheit der Stimmen der anwesenden Mitglieder erforderlich. Wird diese Mehrheit in zwei Wahlgängen nicht erreicht, so ist gewählt, wer im dritten Wahlgang die meisten Stimmen erhalten hat.
  \item Der Fachschaftsvorsitzende verliert das Amt vor Ablauf der Amtszeit durch Neuwahl eines Fachschaftsvorsitzenden mit der Mehrheit der Stimmen der anwesenden Mitglieder der Fachschaftsvertretung, durch Ausscheiden aus der Fachschaftsvertretung oder durch Rücktritt aus wichtigem Grund. Der Rücktritt ist schriftlich gegenüber den anderen Mitgliedern der Fachschaftsvertretung zu erklären.
\end{enumerate}

\subsection{Konstituierende Sitzung}
Die erste Fachschaftsvertretungssitzung der jeweiligen Amtsperiode wird von dem mit den höchsten Stimmzahlen gewählten Mitglied der Fachschaftsvertretung unverzüglich nach Beginn der Amtszeit einberufen. Dieses Mitglied leitet die Sitzung, bis die Wahl des Fachschaftsvorsitzenden abgeschlossen ist.
