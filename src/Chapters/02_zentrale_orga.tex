% !TEX root = ../Main.tex
% ==============================================================================
% Zweiter Abschnitt: Zentrale Organisation
% ==============================================================================

\newpage
\chapter{Zentrale Organisation}
\vspace*{-1.5em}
\rule{\textwidth}{0.5mm}
\vspace*{-3em}

\setcounter{section}{1}
\section*{Erster Unterabschnitt: Das Studierendenparlament (StuPa)}
\addcontentsline{toc}{section}{\textbf{Das Studierendenparlament (StuPa)}}
\sectionmark{Das Studierendenparlament (StuPa)}
\label{sec:zentrale_orga:stupa}

\subsection{Aufgaben}
\label{subsec:zentrale_orga:stupa:aufgaben}
\label{subsec:para:12}
Das StuPa entscheidet über die grundsätzlichen Angelegenheiten der Studierendenschaft einschließlich der Satzungen. Es hat insbesondere folgende Aufgaben:
\begin{enumerate}[label=(\arabic*)]
  \item Wahl, Abberufung und Kontrolle der Mitglieder des AStA,
  \item Verabschiedung der Geschäftsordnung des Studierendenparlaments,
  \item Verabschiedung des Haushaltsplans,
  \item Beratung und Beschlussfassung über alle Satzungen und Ordnungen der Studierendenschaft,
  \item Einsetzen von Referaten und Arbeitskreisen,
  \item Beschlussfassung über Beschwerden von Studierenden, die vorher von der Fachschaft oder dem AStA zurückgewiesen wurden oder direkt bei dem StuPa eingelegt werden,
  \item Beschlussfassung über den Zusammenschluss mit Studierendenschaft anderer Hochschulen zu einem Verband.
\end{enumerate}

\subsection{Zusammensetzung des Studierendenparlaments}
\label{subsec:zentrale_orga:stupa:zusammensetzung}
\label{subsec:para:13}
Das Studierendenparlament setzt sich aus insgesamt \hl{bis zu 21} stimmberechtigten Mitgliedern der Studierendenschaft im Sinne von \cref{subsec:para:1} zusammen. Fünf Mitglieder werden direkt gewählt. Die studentischen Fakultätsräte jeder Fakultät wählen aus ihrer Mitte zwei weitere Mitglieder nach Maßgabe von \cref{subsec:para:10} Abs. 2 \hlrem. Hinzu treten die vier studentischen Senatsmitglieder als Mitglieder kraft \hl{ihres} Amtes. Doppelfunktionen sind nicht zulässig. Jedes Mitglied hat eine Stimme.

\subsection{Vorsitzender des Studierendenparlaments und des Allgemeinen Studierendenausschusses}
\label{subsec:zentrale_orga:stupa:vorsitzender}
\label{subsec:para:14}
\begin{enumerate}[label=(\arabic*)]
  \item Der Vorsitzende des Studierendenparlaments wird mit einfacher Mehrheit der Mitglieder des Studierendenparlaments aus der Mitte des StuPa gewählt. Er ist zugleich Vorsitzender des Allgemeinen Studierendenausschusses.
  \item Der Vorsitzende vertritt die Studierendenschaft nach innen und nach außen.
  \item Der Vorsitzende ist für die Vor- und Nachbereitung sowie die ordnungsgemäße Durchführung der Sitzungen des AStAs sowie des StuPas verantwortlich.
  \item Der Vorsitzende wird vom Finanzreferenten des AStA vertreten, wenn er verhindert ist oder sich zeitweilig ablösen lassen muss. Entsprechend vorherigem Satz vertritt der Schriftführer den Finanzreferenten.
  \item Der Vorsitzende wirkt auf die einheitliche Wahrnehmung der Aufgaben der Studierendenschaft hin, koordiniert die Arbeit des AStA und überwacht die Durchführung der Beschlüsse des AStA.
  \item Der Vorsitzende leitet die zentrale Verwaltung der Studierendenschaft und übt die Weisungsbefugnis gegenüber den Bediensteten der Studierendenschaft aus.
  \item Der Vorsitzende erstattet dem Studierendenparlament über die Arbeit des AStAs sowie dem AStA über die Arbeit des Studierendenparlaments Bericht.
\end{enumerate}

\subsection{Ausscheiden von Parlamentsmitgliedern}
\label{subsec:zentrale_orga:stupa:ausscheiden}
\label{subsec:para:15}
\begin{enumerate}[label=(\arabic*)]
  \item Scheidet ein von den studentischen Fakultätsräten entsandtes Mitglied des Studierendenparlaments aus oder stirbt es, so rückt ein des jeweiligen studentischen Fakultätsrats gewähltes Ersatzmitglied als ständiges Mitglied nach. Die studentischen Fakultätsräte einer jeder Fakultät haben im Weiteren für den Bestand von zwei Ersatzmitgliedern Sorge zu tragen.
  \item Ein von der Fachschaftsvertretung entsandtes Mitglied des Studierendenparlaments scheidet aus dem StuPa aus
  \begin{enumerate}[label=\alph*)]
    \item mit Ablauf der Amtszeit,
    \item durch Exmatrikulation,
    \item durch Rücktritt aus wichtigem Grund, der dem Vorsitzenden der Studierendenschaft gegenüber schriftlich zu erklären ist,
  \end{enumerate}
  \item Ein Mitglied kraft Amtes (studentisches Senatsmitglied) scheidet aus, wenn es sein Amt als studentisches Senatsmitglied verliert. Der/die Nachfolger/in im Amt rückt in das Studierendenparlament ein.
  \item Scheidet ein direkt gewähltes Mitglied des Studierendenparlaments aus, so rückt die Person mit der nächsthöheren Stimmenzahl der entsprechenden Mitgliedschaft nach.
  \item Ein direkt gewähltes Mitglied des Studierendenparlaments scheidet aus dem Parlament aus
  \begin{enumerate}[label=\alph*)]
    \item mit Ablauf der Amtszeit,
    \item durch Exmatrikulation,
    \item durch Rücktritt, der dem Vorsitzenden der Studierendenschaft gegenüber schriftlich zu erklären ist,
    \item durch Tod.
  \end{enumerate}
\end{enumerate}

\subsection{Sitzungen des Studierendenparlaments}
\label{subsec:zentrale_orga:stupa:sitzungen}
\label{subsec:para:16}
\begin{enumerate}[label=(\arabic*)]
  \item Zu der ersten Sitzung des Studierendenparlaments lädt das lebensälteste Mitglied des Studierendenparlaments ein. Es leitet die Sitzung bis die Wahl zum Vorsitzenden der Studierendenschaft abgeschlossen ist.
  \item Ordentliche Sitzungen des Studierendenparlaments sollen in der Vorlesungszeit mindestens einmal monatlich abgehalten werden.
  \item Auf Verlangen des Allgemeinen Studierendenausschusses oder auf Verlangen von mindestens 20 \% der Mitglieder des Studierendenparlaments oder auf Antrag von mindestens 5 \% der gesamten Studierendenschaft finden außerordentliche Sitzungen des Studierendenparlaments statt.
  \item Dem Schriftführer obliegt die Anfertigung und Veröffentlichung des Protokolls. Bei seiner Verhinderung bestimmt zu Sitzungsbeginn der Vorsitzende einen Protokollführer. Die Niederschrift ist vom Schriftführer zu unterzeichnen und in der nächsten Sitzung des StuPas zu genehmigen.
\end{enumerate}

\subsection{Ausschüsse}
\label{subsec:zentrale_orga:stupa:ausschuesse}
\label{subsec:para:17}
Das StuPa kann beratende Ausschüsse einsetzen, die dem Studierendenparlament für ihre Tätigkeit verantwortlich sind. Den Ausschüssen können auch Nichtmitglieder des StuPa mit Sitz und Stimme angehören. Die Mitgliedschaft muss schriftlich dem AStA mitgeteilt werden. \cref{subsec:para:5} Absatz 4 gilt entsprechend. Die Mehrheit der Ausschussmitglieder soll dem Studierendenparlament angehören. Als ständiger Ausschuss wird der Haushaltsausschuss eingerichtet.

\setcounter{section}{2}
\section*{Zweiter Unterabschnitt: Der Allgemeine Studierendenausschuss (AStA)}
\addcontentsline{toc}{section}{\textbf{Der Allgemeine Studierendenausschuss (AStA)}}
\sectionmark{Der Allgemeine Studierendenausschuss (AStA)}
\label{sec:zentrale_orga:asta}

\subsection{Zusammensetzung des AStA}
\label{subsec:zentrale_orga:asta:zusammensetzung}
\label{subsec:para:18}
\begin{enumerate}[label=(\arabic*)]
  \item Die Mitglieder des AStA müssen Mitglieder der Studierendenschaft im Sinne von \cref{subsec:para:1} sein.
  \item Der AStA setzt sich zusammen aus:
  \begin{enumerate}[label=\arabic*.]
    \item dem Vorsitzenden der Studierendenschaft,
    \item dem Finanzreferenten, der zugleich 1. Stellvertreter des Vorsitzenden ist,
    \item einem Schriftführer, der zugleich 2. Stellvertreter des Vorsitzenden ist,
    \item zwei weiteren Referenten (sog. Referatsleiter).
  \end{enumerate}
  Die nähere Aufgaben- und Zuständigkeitsverteilung kann der AStA nach Amtsantritt in seiner Geschäftsordnung regeln, sonst gilt die allgemeine Geschäftsordnung des StuPa entsprechend.
  \item Referatsleiter können sich in ihrer Tätigkeit von freiwilligen Studierenden unterstützen lassen und dazu auch einen Arbeitskreis berufen. Sie berichten dem Studierendenparlament darüber.
\end{enumerate}

\subsection{Aufgaben des AStA}
\label{subsec:zentrale_orga:asta:aufgaben}
\label{subsec:para:19}
\begin{enumerate}[label=(\arabic*)]
  \item Der AStA führt die laufenden Geschäfte der Studierendenschaft.
  \item Der AStA stellt unter Leitung des 1. Stellvertreters einen Finanzplan für ein Haushaltsjahr gemäß den gesetzlichen Vorgaben auf.
  \item Bei unaufschiebbaren Angelegenheiten entscheidet der Vorsitzende anstelle des AStA. Er hat in diesem Fall den AStA unverzüglich zu unterrichten. Der AStA kann die getroffene Entscheidung aufheben, soweit durch ihre Ausführung nicht Rechte Dritter entstanden sind.
  \item Zur Unterstützung des Vorsitzenden bestellt der AStA einen Beauftragten für den Haushalt im Sinne des § 9 LHO, der die Befähigung für den gehobenen Verwaltungsdienst hat oder in vergleichbarer Weise über nachgewiesene Fachkenntnisse im Haushaltsrecht verfügt. Der Haushaltsbeauftragte ist unmittelbar dem Vorsitzenden unterstellt; der Vorsitzende gilt als Leiter der Dienststelle im Sinne des § 9 Abs. 1 S. 2 LHO. Der Finanzreferent arbeitet eng mit dem Beauftragten für den Haushalt zusammen. Erhebt der Haushaltsbeauftragte Widerspruch gegen eine Maßnahme, weil er sie für rechtswidrig oder nach den Grundsätzen der Wirtschaftlichkeit für nicht vertretbar hält, hat der Vorsitzende eine Entscheidung des Studierendenparlaments herbeizuführen.
\end{enumerate}

\subsection{Wahl und Abwahl der Mitglieder des AStA}
\label{subsec:zentrale_orga:asta:mitglieder}
\label{subsec:para:20}
\begin{enumerate}[label=(\arabic*)]
  \item Jedes Mitglied der Studierendenschaft kann sich selbstständig zur Wahl des AStA aufstellen und muss sich bei der ersten Sitzung entsprechender Amtsperiode der Mitglieder des StuPa persönlich vorstellen.
  \item Der Vorsitzende des AStA wird gemäß \cref{subsec:para:14} gewählt.
  \item Die übrigen Mitglieder des AStA werden nach der Wahl des Vorsitzenden ebenfalls mit einfacher Mehrheit der Mitglieder des Studierendenparlaments gewählt. \cref{subsec:para:10} Absatz 1 und 2 gilt entsprechend. Bei mehreren Kandidaten wird jedes Amt einzeln abgestimmt.
  \item Mitglieder des AStA können mit Zweidrittelmehrheit der Mitglieder des Studierendenparlaments abgewählt werden. Ein Mitglied des AStA kann nur abgewählt werden, indem ein neues Mitglied mit Zweidrittelmehrheit der Mitglieder des Studierendenparlaments gewählt wird. Zu der Sitzung, in der die Abwahl erfolgt, muss mindestens zwei Wochen vor dem Termin eingeladen werden.
\end{enumerate}
